\newpage

\begin{thebibliography}{99}
    \bibitem{1} 陈抗生, 周金芳 模拟电路基础: 从系统级到电路级[M],2018.
    \bibitem{2}  陈铖颖, 尹飞飞, 范军 CMOS模拟集成电路设计与仿真实例————基于Hspice[M],北京:电子工业出版社,2014
    \bibitem{3}  包晴晴. 一种高精度流水线ADC数字后台校准技术[D].电子科技大学,2019.
    \bibitem{4}  王莉莎. 12位多路时域交织A/D转换器关键技术研究[D].西安电子科技大学,2018.
    \bibitem{5}  史帅帅. 一种应用在5G通信基站的高速Pipelined ADC设计[D].电子科技大学,2018.
    \bibitem{6}  胡进. 14位250MS/s低功耗无采保流水线A/D转换器设计研究[D].西安电子科技大学,2017.
    \bibitem{7}  万富强. 高精度低功耗流水线型CMOS模数转换器的设计[D].中国科学技术大学,2016.
    \bibitem{8}  田征. SHA-less型流水线ADC的误差分析及校准技术研究[D].西安电子科技大学,2015.
    \bibitem{9}  范超杰. 高性能流水线型模数转换器设计方法研究[D].上海交通大学,2014.
    \bibitem{10}  李天柱. 16位100 MS/s无采保流水线ADC的研究与设计[D].电子科技大学,2014.
    \bibitem{11}  邓世杰. 14bit 250MSPS流水线ADC关键电路设计研究[D].西安电子科技大学,2014.
    \bibitem{12}  张少真. 高速低功耗比较器研究[D].北京交通大学,2012.
    \bibitem{13}  游恒果. 高速低功耗比较器设计[D].西安电子科技大学,2011.
    \bibitem{14}  何伟雄. 低功耗流水线模数转换器的研究与设计[D].电子科技大学,2010.
    \bibitem{15}  过瑶. 低电压低功耗10比特40兆赫兹流水线模数转换器的设计[D].复旦大学,2009.
    \bibitem{16}  修丽梅. 高速低功耗电压比较器结构设计[D].北京交通大学,2008.
    \bibitem{17}  赵周. 基于SoC低功耗Pipeline ADC的设计[D].电子科技大学,2014.
    \bibitem{18}  Hongmei Chen,Lin He,Honghui Deng,Yongsheng Yin,Fujiang Lin. A high-performance bootstrap switch for low voltage switched-capacitor circuits[P]. Radio-Frequency Integration Technology (RFIT), 2014 IEEE International Symposium on,2014.
    \bibitem{19}  Byung-Geun Lee, Byung-Moo Min, Manganaro, G., Valvano, J.W.. A 14b 100MS/s Pipelined ADC with a Merged Active S/H and First MDAC[P]. Solid-State Circuits Conference, 2008. ISSCC 2008. Digest of Technical Papers. IEEE International,2008.
    \bibitem{20}  Texas Instruments, "Dual-channel, RF-sampling AFE with 14-bit, 9-GSPS DACs and 14-bit, 3-GSPS ADCs" AFE7422 datasheet, Oct. 2018 [Revised Jan. 2019]. 
    \bibitem{21}  Texas Instruments, "Quad-Channel, 14-Bit, 1-GSPS, 2x Oversampling, Analog-to-Digital Converter" ADS54J64 datasheet, Oct. 2017. 
    \bibitem{22}  Texas Instruments, "3-GSPS Telecom Receiver and Feedback Device" ADC31RF80 datasheet, Aug. 2017. 

\end{thebibliography}
